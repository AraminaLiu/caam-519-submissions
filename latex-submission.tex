\documentclass{article}

% Language setting
% Replace `english' with e.g. `spanish' to change the document language
\usepackage[english]{babel}

% Set page size and margins
% Replace `letterpaper' with`a4paper' for UK/EU standard size
\usepackage[letterpaper,top=2cm,bottom=2cm,left=3cm,right=3cm,marginparwidth=1.75cm]{geometry}

% Useful packages
\usepackage{amsmath}
\usepackage{graphicx}
\usepackage[colorlinks=true, allcolors=blue]{hyperref}

\usepackage[T1]{fontenc}
\usepackage{inconsolata}


\title{CAAM 519, Homework \# 2: \LaTeX \Submission}
\author{\texttt{xl106}}


\date{September 30, 2021}

\begin{document}
\maketitle


\section{Communicating with remote machines via ssh}
1. the ssh command I used to log into the clear machine
%Use Verbatim Environment
\begin{verbatim}
    ssh xl106@ssh.clear.rice.edu
\end{verbatim}

2. The message sent to me is:
\begin{verbatim}
    Warning: Permanently added the ECDSA host key for IP address '128.42.124.178' to the list of known hosts.
The Rice University Network - Unauthorized access is prohibited
xl106@ssh.clear.rice.edu's password: 
The Rice University Network
 ===========================
 Unauthorized use is prohibited.
 
 This computer system is for authorized users only.  Individuals using this
 system without authority or in excess of their authority are subject to
 having all their activities on this system monitored and recorded or
 examined by any authorized person, including law enforcement, as system
 personnel deem appropriate.  In the course of monitoring individuals
 improperly using the system or in the course of system maintenance, the
 activities of authorized users may also be monitored and recorded.  Any
 material so recorded may be disclosed as appropriate.  Anyone using this
 system consents to these terms.
 
 Problems and/or questions should be submitted via the problem tracking
 system form: http://helpdesk.rice.edu
 
CURRENT USAGE AND LOAD ON THE COMPUTE NODES:
  Fri Oct 15 00:50:01 CDT 2021

 System                   	# Users   	Load ( 5, 10, 15 minute)      
   agate.clear.rice.edu   	    2     	  0.39, 0.28, 0.28            
   amber.clear.rice.edu   	    3     	  0.00, 0.02, 0.05            
   cobalt.clear.rice.edu  	    0     	  0.05, 0.03, 0.05            
   jade.clear.rice.edu    	    2     	  0.05, 0.06, 0.05            
   onyx.clear.rice.edu    	    1     	  0.02, 0.09, 0.12            
   opal.clear.rice.edu    	    4     	  0.09, 0.05, 0.09            
   pyrite.clear.rice.edu  	    2     	  0.01, 0.03, 0.05            

NOTE: !!!!!!!!!!!!!!!!!!!!!!!!!!!!!!!!!!!!!!!!!!!!!!!!!!!!!!!!!!!!!!!!!!!
 
   Please log an RT ticket for any issues you may have.
 
   Please log a ticket at https://help.rice.edu if you have any of the following:
     Feel documentation is lacking.
     Have trouble getting into the system.
     Feel the system is missing a tool.
 
   NOTE: svn.rice.edu is now READ ONLY in preparation for removal 
         from service on January 8th, 2022
 
   NOTE: If you don't have a home drive when using Clear, please 
         create a ticket in help.rice.edu  Be sure to put 
         "need clear home directory" in the subject and one will 
         be setup for you.
 
   NOTE: New tools available.  Maven 3.8.2, gcc 9.3.1, java 11, clang 11.0.1,
         julia 1.6.1, python 3.9.1, vim 8.2 w/ python support.  
         Plus special purpose applications: 
           Circuit Design (ngspice, magic), Network Simulation (ns-3).
         See kb.rice.edu for details.
 
   CLEAR NEWS -- https://kb.rice.edu/internal/page.php?id=71856
   Tips and Hints -- https://kb.rice.edu/internal/page.php?id=71857
 

\end{verbatim}

3. the output of the following shell commands: echo $HOSTNAME

\begin{verbatim}
    [xl106@opal ~]$ echo $HOSTNAME
    opal.clear.rice.edu

\end{verbatim}

4. Also command you use to change your shell prompt text to > In my Ubuntu running in VirtualBox



\section{A script to build a latex document while hiding auxiliary files}
The programming script is as the following, in addition to comments I added before, I add more descriptions for each line of my code, the code is as the following:
\begin{lstlisting}
        #!/bin/bash \\
        #if the directory is not duild file, make the directory with .build\\
        if [ ! -d ".build" ]; then\\
            mkdir .build\\
        fi\\

        #create the project.tex file\\
        pdflatex -output-directory=.build $1.tex\\
        #move the pdf file\\
        mv .build/$1.pdf .\\


\end{lstlisting}
Possible concerns about using this bash script:\\
1. I used $~/Downloads$ and $~/Downloads/.build$. Thus, when downloading clean-build.sh file, I think it is better to use it at $~/Downloads$ directory, which will decrease a lot of unnecessary troubles.\\
2. If the project name typed in the terminal contains spaces between different words or the $"."$ in the file, $pdflatex$ may fail to compile the $.tex$ file. \\

\end{document}